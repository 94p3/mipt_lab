\documentclass[12pt]{kiarticle} % You can learn about my document class "kiarticle" and install it to your device by following the link: https://github.com/Kiarendil/toolkitex
\graphicspath{{pictures/}}
\DeclareGraphicsExtensions{.pdf,.png,.jpg,.eps}
%%%
\pagestyle{fancy}
\fancyhf{}
%\renewcommand{\headrulewidth}{ 0.1mm }
\renewcommand{\footrulewidth}{ .0em }
\fancyfoot[C]{\texttt{\textemdash~\thepage~\textemdash}}
\fancyhead[L]{Лабораторная работа №10.4 \hfil}
\fancyhead[R]{\hfil Иванов Кирилл, 625 группа }
\usepackage{multirow} % Слияние строк в таблице
\newcommand
{\un}[1]
{\ensuremath{\text{#1}}}
\newcommand{\eds}{\ensuremath{ \mathscr{E}}}
\newcommand{\ga}{\ensuremath{\gamma}}
\usepackage{tikz}
%%% Работа с таблицами
\usepackage{array,tabularx,tabulary,booktabs} % Дополнительная работа с таблицами
\usepackage{longtable}  % Длинные таблицы
\usepackage{multirow} % Слияние строк в таблице

\begin{document}
	
	\begin{titlepage}
		\begin{center}
			\large 	Московский физико-технический институт \\
			(государственный университет) \\
			Факультет общей и прикладной физики \\
			\vspace{0.2cm}
			
			\vspace{4.5cm}
			Лабораторная работа № 10.4 \\ \vspace{0.2cm}
			\large (Общая физика: квантовая физика) \\ \vspace{0.2cm}
			\LARGE \textbf{ Магнитные моменты легких ядер }
		\end{center}
		\vspace{2.3cm} \large
		
		\begin{center}
			Работу выполнил: \\
			Иванов Кирилл,
			625 группа
			\vspace{10mm}		
			
		\end{center}
		
		\begin{center} \vspace{60mm}
			г. Долгопрудный \\
			2018 год
		\end{center}
	\end{titlepage}


	\paragraph*{Цель работы:} 
	В работе вычисляются магнитные моменты протона, дейтрона и ядра фтора на основе измерения их $ g $-факторов методом ядерного магнитного резонанса (ЯМР). Полученные данные сравниваются с вычислениями магнитных моментов на основе кварковой модели адронов и одночастичной оболочечной модели ядер.
	
	
	\section{Теоретическое введение}
	
	Момент количества движения ядра, который принимает целые для четного числа нуклонов или полуцелые для нечетных (в единицах $ \hbar $) складывается из спина ядра $ S $ и полного орбитального момента нуклонов $ L $:
	
	\begin{equation}\label{}
	I = L + S
	\end{equation} 
	
	При этом у ядер существует магнитный момент $ \mu $, связанный с $ I $. Их отношение называется гиромагнитным отношение $ \gamma = g \gamma_0 $, где $ g $ --- фактор Ланде, или $ g $-фактор, а $ \gamma_0 = -\frac{e}{2m_ec} $. Аналогично  $ \gamma_я = \frac{e}{2m_pc} $
	
	Магнитный момент таким образом можно записать как 
	
	\begin{equation}\label{}
	\mu = \gamma_я \hbar I 
	\end{equation}
	
	Измерять его можно через ядерный магнетрон
	
	\begin{equation}\label{}
	\mu_я = \gamma_я \hbar = \frac{e}{2m_pc} \hbar 
	\end{equation}
	
	Таким образом, запишем магнитный момент в виде:
	
	\begin{equation}\label{mu}
	\mu = \mu_я g_я I
	\end{equation}

	
	В данной работе исследуется ядерный магнитный резонанс (ЯМР). Если пропускать
	атомы сквозь сильное магнитное поле, связь $ I $ и $ J  $разрывается, и
	оба эти вектора независимо прецессируют вокруг $ H $ с угловой частотой
	$ \omega = g(eH/Mc) $, где $ g $ --- гиромагнитное отношение. Если теперь наложить слабое добавочное магнитное поле $ H' $, перпендикулярное к основному полю, то оно вызовет изменение ориентации ядерных спинов.
	Этот эффект может быть обнаружен, так как он оказывает влияние
	на траекторию атомов.
	
	Этот метод может быть применен и к неподвижным ядрам и тогда он называется методом ЯМР. ЯМР --- это резонансное поглощение
	электромагнитной энергии в веществах, обусловленное ядерным перемагничиванием. ЯМР наблюдается в постоянном магнитном поле $ H_0 $ при одновременном воздействии на образец радиочастотного магнитного поля, перпендикулярного $ H_0 $, и обнаруживается по поглощению излучения. 
	
	В магнитном поле ядерные уровни расщепляются (появляется так называемое зеемановское расщепление) и под действием внешнего высокочастотного поля могут происходить переходы между компонентами расщепившегося уровня, это явление носит резонансный характер. Различие по энергии между двумя соседними компонентами определяется формулой
	
	\begin{equation}\label{}
	\Delta E = g_я \mu_я B_0 = hf_0
	\end{equation}
	
	Из условия, что $ \Delta E $ равна энергии квантов, задающие электромагнитные переходы, при резонансной частоте $ f_0 $ можно найти фактор Ланде по формуле:
	
	\begin{equation}\label{g}
	g_{я} = \dfrac{hf_0}{\mu_я B_0} 
	\end{equation}

%	\section{Экспериментальная установка}
	
	\section{Выполнение работы}
	
	Запишем значение $ B_0 $ постоянного магнита:
	
	\begin{equation}\label{}
	B_0 = 142,7 \; мТл
	\end{equation} 
	
	Найдем значения резонансной частоты для трех разных образцов:
	
	\begin{itemize}
		\item \textbf{Вода, ЯМР на ядрах водорода}: $ f_0 = 5,87 \pm 0,01 $ МГц
			
		\item \textbf{Резина, ЯМР на ядрах водорода}: $ f_0 = 5,87 \pm 0,01 $ МГц
			
		\item \textbf{Тефлон, ЯМР на ядрах фтора}: $ f_0 = 5,52 \pm 0,01 $ МГц
	\end{itemize}
	
	По формулам \eqref{g} и \eqref{mu} вычислим значения $ g $-фактора и магнитного момента соответственно ($ I = \frac{1}{2} $ для водорода и фтора): 
	
	\begin{itemize}
		\item \textbf{Вода, ЯМР на ядрах водорода}: $ g_{яH} = 5,398 \pm 0,009, \quad \mu_p  = (2,699 \pm 0,005)\mu_я $ 
		
		\item \textbf{Резина, ЯМР на ядрах водорода}: $ g_{яH} = 5,398 \pm 0,009, \quad \mu_p   = (2,699 \pm 0,005)\mu_я  $ 
		
		\item \textbf{Тефлон, ЯМР на ядрах фтора}: $ g_{яF} = 5,079 \pm 0,009, \quad \mu_F  = (2,539 \pm 0,005)\mu_я  $ 
	\end{itemize}

Для дейтерия, возьмем результат измерения у коллег по группе из-за отсутствия его на нашей установке:

\begin{itemize}
	\item $ f_0 = 0,378 \pm 0,003 $ МГц, $ B = 517 \pm 2 $ мТл, тогда $ g = 0,857 \pm 0,004 \te \mu_d = (0,857 \pm 0,004) \mu_я $ (т.к $ I_d = 1 $).
\end{itemize}
	
	\section{Вывод}
	
	Сведем полученные результаты в таблицу:
	
	
	\begin{table}[H]
		\caption{Итоговые результаты}
		\begin{center}
			\begin{tabular}{|c|c|c|c|c|}
				\hline
				Образец  & $ f_0 $, МГц &$ g_я $  & $ \mu $ (в ед. $ \mu_я $) & $ \mu_{я\; таблич} $ (в ед. $ \mu_я $) \\ \hline
				Вода (ядра $ H $)  & $ 5,87 \pm 0,01 $ &   $ 5,398 \pm 0,009 $   &  $ 2,699 \pm 0,005 $  & 2,793 \\
				Резина (ядра $ H $)  & $ 5,87 \pm 0,01 $ &   $ 5,398 \pm 0,009 $   &  $ 2,699 \pm 0,005 $  & 2,793 \\
				Тефлон (ядра $ F $)  & $ 5,52 \pm 0,01 $ &   $ 5,079 \pm 0,009 $   &  $ 2,539 \pm 0,005 $  & 2,629 \\
				Тяжелая вода (ядра $ ^2H $)  & $ 0,378 \pm 0,003$ &   $ 0,857 \pm 0,004 $   &  $ 0,857 \pm 0,004 $  & 0,857 \\
				 \hline
			\end{tabular}
		\end{center}
		\label{table_5}
	\end{table}

	Видно, что полученные значения достаточно близки к табличным. При этом согласно кварковой модели адронов, $ \mu_p = 3 \mu_я $, и для ядра фтора момент, вычисленный при помощи одночастичной оболочечной модели ядер, равен $ \mu_F = \mu_я $.
	
	Подсчитаем долю состояния $ ^3_1D $ в основном состоянии дейтрона (из предположения о наличии в нем  $ ^3_1S $ и  $ ^3_1D $ состояний):
	
	\begin{equation}\label{}
	P_D = \dfrac{2}{3}\dfrac{\mu_p + \mu_n - \mu_d}{\mu_p + \mu_d - \frac{1}{2}} \approx 0,04
	\end{equation}
	\begin{equation}\label{}
	\oint xdx = \left/  \dfrac{x}{y} \; Damir \; Petuh \right/  = \cos x 
	\end{equation}
\end{document}
