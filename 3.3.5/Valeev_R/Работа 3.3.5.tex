\documentclass[a4paper, 12pt]{article}%тип документа

%отступы
\usepackage[left=2cm,right=2cm,top=2cm,bottom=3cm,bindingoffset=0cm]{geometry}

%Русский язык
\usepackage[T2A]{fontenc} %кодировка
\usepackage[utf8]{inputenc} %кодировка исходного кода
\usepackage[english,russian]{babel} %локализация и переносы

%Вставка картинок
\usepackage{wrapfig}
\usepackage{graphicx}
\graphicspath{{pictures/}}
\DeclareGraphicsExtensions{.pdf,.png,.jpg}

%оглавление
\usepackage{titlesec}
\titlespacing{\chapter}{0pt}{-30pt}{12pt}
\titlespacing{\section}{\parindent}{5mm}{5mm}
\titlespacing{\subsection}{\parindent}{5mm}{5mm}
\usepackage{setspace}

%Графики
\usepackage{multirow}
\usepackage{pgfplots}
\pgfplotsset{compat=1.9}

%Математика
\usepackage{amsmath, amsfonts, amssymb, amsthm, mathtools}

%Заголовок
\author{Валеев Рауф Раушанович \\
группа 825}
\title{\textbf{Работа 3.3.5\\
Эффект Холла в металлах}}
\begin{document}
\begin{table}[]
\begin{tabular}{|c|c|c|c|c|c|c|c|}
\hline
 & \multicolumn{3}{c|}{$(I = 0,4 \pm 0,01)$, А} & \multicolumn{4}{c|}{$U_0 = (2 \pm 1)$ ед.} \\ \hline
 & $I_{\text{м}}$, А & $\sigma_{I_{\text{м}}}, A$ & $\sigma_B$, мТл & $B$, мТл & $U$, ед & $U$, нВ & $\sigma_{U}$, нВ \\ \hline
1 & 0,20 & 0,01 & 130 & 230 & 3 & 40 & 20 \\ \hline
2 & 0,39 & 0,01 & 80 & 450 & 5 & 120 & 20 \\ \hline
3 & 0,60 & 0,01 & 70 & 700 & 7 & 200 & 20 \\ \hline
4 & 0,80 & 0,01 & 65 & 900 & 9 & 280 & 20 \\ \hline
5 & 1,01 & 0,01 & 70 & 1150 & 10 & 320 & 20 \\ \hline
6 & 1,20 & 0,01 & 80 & 1400 & 11 & 360 & 20 \\ \hline
\end{tabular}
\end{table}
\end{document}

