\documentclass[a4paper, 10pt]{article}%тип документа

%Русский язык
\usepackage[T2A]{fontenc} %кодировка
\usepackage[utf8]{inputenc} %кодировка исходного кода
\usepackage[english,russian]{babel} %локализация и переносы

%Вставка картинок
\usepackage{graphicx}
\graphicspath{{pictures/}}
\DeclareGraphicsExtensions{.pdf,.png,.jpg}

%Графики
\usepackage{pgfplots}
\pgfplotsset{compat=1.9}

%Математика
\usepackage{amsmath, amsfonts, amssymb, amsthm, mathtools}

%Заголовок
\author{Валеев Рауф Раушанович \\
группа 825}
\title{Работа 1.1.4 \\
Измерение интенсивности радиационного фона}
\begin{document}
\maketitle
\newpage
\textbf{Теоритическая справка.} Если случайные события (регистрация частиц) однородны во времени и каждое последущее событие не зависит от того, когда и как случилось предыдущее, то такой процесс называется пуассоновским, а результататы - количество отсчётов в одном опыте - подчиняются так называемому распределению Пуассона. При больших числах отсчёт это распределение стремится к нулю. 

\textbf{Цель работы:} применение методов обработки экспериментальных данных для изучения статистических закономерностей при измерении интенсивности радиационного фона.

\textbf{В работе используются:} счетчик Гейгера-Мюллера (СТС-6), блок питания, компьютер с интерфейсом связи с счетчиком.
\begin{enumerate}
\item Включаем питание компьютера и установки. После загрузки компьютера запускаем программу STAT и таким образом начинаем проведение основного эксперимента. 
\item В результате демонстрационного эксперимента убеждаемся, что при увеличении числа измерений:
\begin{enumerate}
\item Измеряемая велечина флуктуирует;
\item Флуктуации среднего значения измеряемой величины уменьшаются, и среднее значение выходит на постоянную величину;
\item Флуктуации велечины погрешности среднего значения уменьшаются, а сама величина убывает;
\item Флуктуации величины погрешности отдельного измерения уменьшаются, и погрешность отдельного измерения (погрешность метода) выходит на постоянную величину.
\end{enumerate}
\item Переходим к основному эксперименту: измерение плотности потока космического излучения за 20 секунд (результаты набрались с момента включения компьютера). На компьютере проведем обработку, аналогичную сделанной в демонстрационном эксперименте. Результаты приведены в табл. 1.
\item Разбиваем результаты из табл. 1 в порядке их получения на группы по 2, что соответствует произведению $N_2 = 100$ измерений числа частиц за интервал времени, равный 40 с. Результаты приведем в табл. 2.
\item Приведем данные ддя построения гистограмм распределения числа срабатываний счетчика за 10 с и 40 с в таблицах табл. 3 и табл. 4 соответственно. 
\item Так же приведем гистограммы распределений среднего числа отсчетов за 10 и 40 с. (Рис. 1) 
\item Используя формулы
\[ \overline{n}_1 = \dfrac{1}{N_1} \sum_{i = 1}^{N_1} {n_i} 
 \]
 \[ \overline{n}_2 = \dfrac{1}{N_2} \sum_{i = 1}^{N_2} {n_i} 
 \]
 \[ \overline{n}_3 = \dfrac{1}{N_3} \sum_{i = 1}^{N_3} {n_i} 
 \] 
\item Определим среднее число срабатываний счетчика за 10, 20 и 40 с соответственно.
\item Найдем среднеквадратичные ошибки отдельных измерений по формулам по формулам
\[ \sigma_1 = \sqrt{\dfrac{1}{N_1 - 1} \sum_{i = 1}^{N_1} {(n_i - \overline{n}_1)^2} }
 \]
 \[ \sigma_2 = \sqrt{\dfrac{1}{N_2 - 1} \sum_{i = 1}^{N_2} {(n_i - \overline{n}_2)^2} }
 \]
 \[ \sigma_3 = \sqrt{\dfrac{1}{N_3 - 1} \sum_{i = 1}^{N_3} {(n_i - \overline{n}_3)^2} }
 \]
и убедимся в справедливости формул
 \[\sigma_1 \approx\sqrt{\overline{n}_1} \]
 \[\sigma_2 \approx\sqrt{\overline{n}_2} \]
 \[\sigma_3 \approx\sqrt{\overline{n}_3} \]
 \item найдем ошибки всех измерений по формулам
 \[ \sigma_1 = \sqrt{\dfrac{1}{(N_1 - 1)N_1} \sum_{i = 1}^{N_1} {(n_i - \overline{n}_1)^2} }
 \]
 \[ \sigma_2 = \sqrt{\dfrac{1}{(N_2 - 1)N_2} \sum_{i = 1}^{N_2} {(n_i - \overline{n}_2)^2} }
 \]
 \[ \sigma_3 = \sqrt{\dfrac{1}{(N_3 - 1)N_3} \sum_{i = 1}^{N_3} {(n_i - \overline{n}_3)^2} }
 \]
\item Зафикисруем все полученные ошибки и среднии значения срабатываний в табл. 5.
\item Определим долю случаев, когда отклонения не превышают $\sigma_i$ и $2\sigma_i$, и сравним с теоретическими оценками (табл. 6).
\item Посчитаем относительную ошибку по формуле
\[ \varepsilon_{\overline{n}_1} = \dfrac{\sigma_{\overline{n}_1}}{\overline{n}_1} 100 \% \]
\[ \varepsilon_{\overline{n}_2} = \dfrac{\sigma_{\overline{n}_2}}{\overline{n}_2} 100 \% \]
\[ \varepsilon_{\overline{n}_3} = \dfrac{\sigma_{\overline{n}_3}}{\overline{n}_3} 100 \% \]
\item из табл. 6 следует, что $n_{t=10 c} = 13,32 \pm 0,18, \varepsilon_{\overline{n}_1} = 1,4 \%$, $n_{t=20 c} = 26,6 \pm 0,36, \varepsilon_{\overline{n}_1} = 1,4 \%$, $n_{t=40 c} = 53,11 \pm 0,61, \varepsilon_{\overline{n}_1} = 1,2 \%$
\end{enumerate}
\begin{tikzpicture}
\begin{axis}[
	legend pos = north west,
	height = 0.6\paperheight, 
	width = 0.5\paperwidth,
	ybar,
	title = Рис. 1,
	xlabel = {$\, \, \, \, \, \, \,39 \; \; \; \; 42 \; \; \; \; \, 46 \; \; \; \; \, 49 \; \; \; \; \, 52 \; \; \; \; 56 \; \; \; \;  59 \; \; \; \; 62 \; \; \; \; 66 \; \; \; \;  69 \; \; \; \; 73 \, \, \, \,$},
	ylabel = {$\omega$}
	]
\legend{ 
	$10 c$, 
	$40 c$
};
\addplot coordinates { 
(6, 0.01) (7, 0.045) (8, 0.045) (9, 0.06) (10, 0.085) (11, 0.075) (12, 0.1025) (13, 0.0975) (14, 0.11) (15, 0.0825) (16, 0.085) (17, 0.0775) (18, 0.0475) (19, 0.03) (20, 0.02) (21, 0.02) (22, 0.0025) (23, 0) (24, 0.005) (25, 0)
 };
 \addplot coordinates { 
(6,0.02) (6.57,0) (7.14,0.01) (7.71,0) (8.28,0.01) (8.85,0.02) (9.42,0.01) (9.99,0.01) (10.56,0.09) (11.13,0.07) (11.7,0.07) (12.27,0.1) (12.84,0.02) (13.41,0.03) (13.98,0.04) (14.55,0.07) (15.12,0.06) (15.69,0.15) (16.26,0.03) (16.83,0.02) (17.4,0.02) (17.97,0.03) (18.54,0.03) (19.11,0.04) (19.68,0.03) (20.25,0) (20.82,0)(21.39,0)(21.96,0)(22.53,0)(23.1,0)(23.67,0) (24.24,0) (24.81,0.01) (25.38,0.01)
 };
\end{axis}
\end{tikzpicture}\\
\newpage
\begin{table}
\caption{\textbf{Число срабатываний счетчика за 20 с}}
\begin{tabular}{|r|c|c|c|c|c|c|c|c|c|c|}
\hline
№ опыта&1&2&3&4&5&6&7&8&9&10\\
\hline
0&29&21&33&27&22&26&26&27&28&27\\
\hline
10&26&22&25&22&25&31&24&20&23&31\\
\hline
20&31&31&20&19&29&27&25&31&35&21\\
\hline
30&38&25&27&29&30&31&36&23&28&28\\
\hline
40&25&25&29&20&15&26&24&30&25&23\\
\hline
50&21&29&17&32&32&15&29&29&21&27\\
\hline
60&21&22&35&26&23&26&29&19&18&31\\
\hline
70&25&31&36&22&27&29&23&26&29&25\\
\hline
80&22&25&31&33&19&31&23&28&29&27\\
\hline
90&32&22&23&33&22&25&29&26&23&24\\
\hline
100&21&35&27&35&27&25&18&29&28&32\\
\hline
110&33&39&29&30&25&28&23&30&22&25\\
\hline
120&21&23&34&22&19&29&20&35&17&22\\
\hline
130&21&33&19&31&23&28&44&29&23&24\\
\hline
140&32&25&26&26&27&26&21&24&29&34\\
\hline
150&34&27&24&31&23&39&22&35&23&32\\
\hline
160&29&17&28&27&20&30&27&29&28&19\\
\hline
170&38&25&23&25&25&25&27&23&28&24\\
\hline
180&21&28&27&23&30&19&32&22&21&33\\
\hline
190&26&30&28&29&28&34&22&38&21&29\\
\hline
\end{tabular}
\caption{\textbf{Число срабатываний счетчика за 40 с}}
\begin{tabular}{|r|c|c|c|c|c|c|c|c|c|c|}
\hline
№ опыта&1&2&3&4&5&6&7&8&9&10\\
\hline
0&50&60&48&53&55&48&47&56&44&54\\
\hline
10&62&39&56&56&56&63&56&61&59&56\\
\hline
20&50&49&41&54&48&50&49&47&58&48\\
\hline
30&43&61&49&48&49&56&58&56&49&54\\
\hline
40&47&56&50&51&56&54&56&47&55&47\\
\hline
50&56&62&52&47&60&72&59&53&53&47\\
\hline
60&44&56&48&55&39&54&50&51&73&47\\
\hline
70&57&52&53&45&63&61&55&62&57&55\\
\hline
80&46&55&50&56&47&63&48&50&50&52\\
\hline
90&49&50&49&54&54&56&57&62&60&50\\
\hline
\end{tabular}
\end{table}
\begin{table}
\caption{\textbf{Данные для построения гистограммы распределения числа срабатываний счетчиков за 10 с}}
\begin{tabular}{|r|p{1cm}|p{1cm}|p{1cm}|p{1cm}|p{1cm}|}
\hline
Число импульсов $n_i$&6&7&8&9&10\\
\hline
Число случаев&4&18&18&24&34\\
\hline
Доля случаев $\omega_n$&0,01&0,045&0,045&0,06&0,085\\
\hline
\end{tabular}
\begin{tabular}{|r|p{1cm}|p{1cm}|p{1cm}|p{1cm}|p{1cm}|}
\hline
Число импульсов $n_i$&16&17&18&19&20\\
\hline
Число случаев&34&31&19&12&8\\
\hline
Доля случаев $\omega_n$&0,085&0,0775&0,0475&0,03&0,02\\
\hline
\end{tabular}
\begin{tabular}{|r|p{1cm}|p{1cm}|p{1cm}|p{1cm}|p{1cm}|}
\hline
Число импульсов $n_i$&21&22&23&24&25\\
\hline
Число случаев&8&1&0&2&0\\
\hline
Доля случаев $\omega_n$&0,02&0,0025&0&0,005&0\\
\hline
\end{tabular}
\end{table}
\begin{table}
\caption{\textbf{Данные для построения гистограммы распределения числа срабатываний счетчиков за 40 с}}
\begin{tabular}{|r|p{1cm}|p{1cm}|p{1cm}|p{1cm}|p{1cm}|p{1cm}|p{1cm}|}
\hline
Число импульсов $n_i$&39&40&41&42&43&44&45\\ 
\hline
Число случаев&2&0&1&0&1&2&1\\ 
\hline
Доля случаев$\omega_n$&0,02&0&0,01&0&0,01&0,02&0,01\\ 
\hline
\end{tabular}
\begin{tabular}{|r|p{1cm}|p{1cm}|p{1cm}|p{1cm}|p{1cm}|p{1cm}|p{1cm}|}
\hline
Число импульсов $n_i$&46&47&48&49&50&51&52\\ 
\hline
Число случаев&1&9&7&7&10&2&3\\ 
\hline
Доля случаев $\omega_n$&0,01&0,09&0,07&0,07&0,1&0,02&0,03\\ 
\hline
\end{tabular}
\begin{tabular}{|r|p{1cm}|p{1cm}|p{1cm}|p{1cm}|p{1cm}|p{1cm}|p{1cm}|}
\hline
Число импульсов $n_i$&53&54&55&56&57&58&59\\ 
\hline
Число случаев&4&7&6&15&3&2&2\\ 
\hline
Доля случаев $\omega_n$&0,04&0,07&0,06&0,15&0,03&0,02&0,02\\
\hline 
\end{tabular}
\begin{tabular}{|r|p{1cm}|p{1cm}|p{1cm}|p{1cm}|p{1cm}|p{1cm}|p{1cm}|}
\hline
Число импульсов $n_i$&60&61&62&63&64&65&66\\ 
\hline
Число случаев&3&3&4&3&0&0&0\\ 
\hline
Доля случаев $\omega_n$&0,03&0,03&0,04&0,03&0&0&0\\
\hline
\end{tabular}
\begin{tabular}{|r|p{1cm}|p{1cm}|p{1cm}|p{1cm}|p{1cm}|p{1cm}|p{1cm}|}
\hline
Число импульсов $n_i$&67&68&69&70&71&72&73\\
\hline
Число случаев&0&0&0&0&0&1&1\\
\hline
Доля случаев $\omega_n$&0&0&0&0&0&0,01&0,01\\
\hline
\end{tabular}
\end{table}
\begin{table}
\caption{\textbf{Ошибки и средние значения}}
\begin{tabular}{|c|c|c|c|c|}
\hline
&$\overline{n}$&$\sigma_{\text{среднеквадратичная}}$&$\sigma_{\text{примерная}}$&$\sigma_{\text{общая}}$\\
\hline
1&13,32&3,65&3,63&0,18\\
\hline
2&26,6&5,16&5,03&0,36\\
\hline
3&53,11&6,1&7,29&0,61\\
\hline
\end{tabular}
\caption{\textbf{Процент попадания точек в промежуток среднего значения с учетом погрешности}}
\begin{tabular}{|c|c|c|c|c|}
\hline
Значение&Ошибка&Число случаев&Доля случаев,$\%$&Теоретическая оценка,$\%$ \\
\hline
$\overline{n}_1 = 13.32$ & $\pm \sigma_1 = \pm 3,65$ & 286 & 71 & 68 \\
& $\pm 2 \sigma_1 = \pm 7,3$ & 389 & 97 & 95 \\
\hline
$\overline{n}_2 = 13.32$ & $\pm  \sigma_2 = \pm 5,16$ & 138 & 69 & 68 \\
& $\pm 2 \sigma_2 = \pm 10,32$ & 192 & 96 & 95 \\
\hline
$\overline{n}_3 = 13.32$ & $\pm  \sigma_3 = \pm 6,1$ & 68 & 68 & 68 \\
& $\pm 2 \sigma_3 = \pm 12,2$ & 96 & 96 & 95 \\
\hline
\end{tabular}
\end{table}
\end{document}
